\documentclass{livret}
\usepackage[latin1]{inputenc}
\usepackage[T1]{fontenc}
\usepackage[a4paper,left=2cm,right=2cm,top=2cm,bottom=2cm]{geometry}
\usepackage[frenchb]{babel}
\usepackage{libertine}
\usepackage[pdftex]{graphicx}
\usepackage{eurosym}


\newcommand{\hsp}{\hspace{20pt}}
\newcommand{\HRule}{\rule{\linewidth}{0.5mm}}

\begin{document}

\begin{titlepage}
  \begin{sffamily}
  \begin{center}
		\includegraphics[width=200pt]{cpe.jpg}

    \textsc{\Large Rapport de projet d'�lectronique}\\[1.5cm]

    % Title
    \HRule \\[0.4cm]
    { \huge \bfseries Centrale inertielle\\[0.4cm] }

    \HRule \\[2cm]

    % Author and supervisor
    \begin{minipage}{0.4\textwidth}
      \begin{flushleft} \large
        MOUGIN \textsc{Paul}\\
        RATTRAPAGE DE 4ETI 2013\\
      \end{flushleft}
    \end{minipage}

    \vfill

    % Bottom of the page
    {\large Juin 2015}

  \end{center}
  \end{sffamily}
\end{titlepage}

\tableofcontents

\chapter*{Introduction}
\addcontentsline{toc}{chapter}{Introduction}
	Le but de ce projet est de cr�er � l'aide d'une carte �lectronique STM32F3 Discovery une centrale inertielle en utilisant les acc�l�rom�tres et les gyroscopes embarqu�s dans la carte.
	
\chapter{Etudes pr�alables}
	\section{Quid d'une centrale inertielle ?}
	\paragraph{}
	
	\section{Consid�rations th�oriques}
	\paragraph{}
	
	\section{Mat�riel mis a disposition}
	\paragraph{}

\chapter{R�alisation du projet}
	\section{Impl�mentation sur la carte STM32F3}
		\paragraph{}

	\section{Interface utilisateur}
	\paragraph{}
	
\chapter{Am�liorations possibles}
	\section{Gestion des acc�l�rom�tres}
	\paragraph{}
	
	\section{Fiabilit�}
	\paragraph{}
		
\chapter*{Conclusion}
\addcontentsline{toc}{chapter}{Conclusion}
\paragraph{}
	
\end{document}